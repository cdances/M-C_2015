\documentclass{mc2015}

%%%%%%%%%%%%%%%%%%%%%%%%%%%%%%%%%%%%%%%%%%%%%%%%%%%%%%%%%%%%%%%%%%%%%
\usepackage[T1]{fontenc}         % Use T1 encoding instead of OT1
\usepackage[utf8]{inputenc}      % Use UTF8 input encoding
\usepackage{microtype}           % Improve typography
\usepackage{booktabs}            % Publication quality tables
\usepackage{amsmath}
\usepackage{graphicx}
\usepackage{float}
\usepackage[exponent-product=\cdot]{siunitx}
\usepackage[colorlinks,breaklinks]{hyperref}
\hypersetup{linkcolor=black, citecolor=black, urlcolor=black}

\usepackage{lipsum}

\def\equationautorefname{Eq.}
\def\figureautorefname{Fig.}

%%%%%%%%%%%%%%%%%%%%%%%%%%%%%%%%%%%%%%%%%%%%%%%%%%%%%%%%%%%%%%%%%%%%%
% Insert authors' names and short version of title in lines below

\authorHead{Chris Dances, Vince Mousseau, Maria Avramova}
\shortTitle{Initial Verification of CTF}

%%%%%%%%%%%%%%%%%%%%%%%%%%%%%%%%%%%%%%%%%%%%%%%%%%%%%%%%%%%%%%%%%%%%%
\begin{document}

\title{Initial 1-D Single Phase Liquid Verification of CTF}

\author{Chris Dances}
\author{Dr. Maria Avramova}
\affil{ Department of Mechanical and Nuclear Engineering \\
  The Pennsylvania State University \\
  137 Reber Building, University Park, PA, 16802, USA \\
  cad39@psu.edu; mna109@psu.edu}

\author{Dr. Vince Mousseau}
\affil{ Computer Science Research Institute \\
  Sandia National Laboratories \\
  1450 Innovation Parkway, Albuquerque, NM 87123, USA \\
  vamouss@sandia.gov
}

\maketitle

\begin{abstract}
Nuclear engineering codes are being used to simulate more challenging problems
and at higher fidelities than for which they were initially developed for. In
order to expand the capabilities of these codes, state of the art numerical methods and
software quality measures need to be implemented. One of the key players in this
effort is the Consortium for Advanced Simulation of Light Water Reactors (CASL) through
development of the Virtual Environment for Reactor Applications (VERA). The
sub-channel thermal hydraulic code used in VERA, COBRA-TF (Coolant-Boiling in
Rod Arrays - Three Fluids), is partially developed at the Pennsylvania State
University by the Reactor Dynamics and Fuel Management Research Group (RDFMG).
The RDFMG of version COBRA-TF is referred to as CTF.

In an effort to help meet the objectives of CASL, a version of CTF has been
developed that solves the residual formulation of the one dimensional 
single-phase conservation equations. The formulation of the base equations as
residuals allows for the isolation of different sources of error and is a
good tool for verification purposes. This paper outlines the initial
verification work of both the original version of CTF and its residual
formulation. The verification problem is a simple 1-D single phase liquid
channel with no heat conduction, friction, or gravity. A transient boundary
condition is applied that alters the inlet density and temperature while keeping
the velocity within the channel constant. The constant velocity simplifies the modified
equation analysis and the order of accuracy is readily obtained. A Richardson
extrapolation is performed on the problem on the temporal and spatial step sizes
to determine the convergence and order of accuracy of the discretization error.
While extensive validation work has been present for CTF, there has been little
to no verification work previously. 

\emph{Key Words}: CTF, thermal hydraulic, verifiation, residual, Richardson
extrapolation, CASL

\end{abstract}

\clearpage
%%%%%%%%%%%%%%%%%%%%%%%%%%%%%%%%%%%%%%%%%%%%%%%%%%%%%%%%%%%%%%%%%%%%%

%\tableofcontents
%\clearpage
%
%\listoffigures
%\clearpage

%%%%%%%%%%%%%%%%%%%%%%%%%%%%%%%%%%%%%%%%%%%%%%%%%%%%%%%%%%%%%%%%%%%%%
\section{Introduction}

For the past several decades, the primary focus in nuclear engineering within
the United States has been on light water reactors (LWR). Commercially,
all nuclear reactors are either boiling water reactors (BWR) or pressurized
water reactors (PWR). Correct computation of the thermal hydraulics within the
reactor core leads to efficient design and accuracy in the safety analysis. A
popular subchannel code for modeling the hydrodynamics within the reactor core
is CTF, which is a subchannel thermal-hydraulics code developed from
COBRA-TF \cite{Salko2014}. This FORTRAN based code solves 8 conservation
equations for liquid, entrained droplet, and vapor phases phases, plus one
conservation equation for non-condensable gases. A 1-D residual formulation of
the code has been created. While other residual formulations have been
formed for other versions of COBRA-TF \cite{Lloyd2014}, none have been
integrated into the CASL version of CTF. The current version of CTF has standard
verification practices that focus on software quality engineering similar to
those in other versions of COBRA-TF \cite{Aumiller2013}, but lacks an in
depth verification document that focuses on numerical algorithm verification.
This paper focues on this second type of verification and outlines the initial
verification of the original version of the code as well as the residual version
of the code. The verification problem is a single phase 1-D channel with
transient inlet density and mass flow rate. The problem will undergo a
Richardson's extrapolation in the temporal and spatial domains to verify the
convergence and order of accuracy of the error. The study of the order of
accuracy is considered one of the more rigorous verification criteria \cite{Roy2005}.
This work will be expanded to perform verificaiton on the single
phase equations in both axial and transverse dimensions \cite{Merroun2009}, and
coupled fluid heat conduction \cite{Mahadevan2009}.

\section{CTF}

The thermal hydraulics of a LWR core is an important part of nuclear
reactor design. CTF has the ability to solve for the temperature and pressure
of water within the rod structure of a LWR reactor core. Currently, the conservation
equations analytically reduce into a pressure matrix in a semi-implicit
method with rod temperatures solved for explicitly. The residual formulation of
the code currently solves the 1-D single phase liquid conservation equations and
calculated variables in a residual formulation. While it has the ability to
solve the conservation equations semi-implicitly or implicitly, only the
semi-implicit solution method is considered in this paper. This residual
formulation should allow for easier and more in depth verification  analysis.
This paper details the initial verificaiton of the residual formulation and
original code.

\subsection{Software Quality Assurance}

Software quality assurance is a set of tools and procedures that help
ensure that the software is reliable. CTF is managed by GitHub repository
setup and maintained by CASL. An extensive test matrix is run before each major
push to ensure that the code meets the specified requirements. The test matrix
consists of a fully automated suite of regression tests that include unit and
validation tests. This paper will be the beginning of a verification manual,
integrating this verification problem directly into the test suite.

\subsection{1-D Single Phase Liquid Conservation Equations}

The finite volume structure in CTF in figure \ref{fig:CTF-Cells} is for a
one-dimensional channel in the axial direction with $n$ number of cells. The
first and last cells at 0 and $n+1$ are ghost cells and act as the boundary
conditions for the problem. Pressure, enthalpy, and density are averaged over
the cell volume and are located at the center of the cell. Mass flow rate and
velocity are located at the faces in between cells. The cells  are represented
with an index $i$, and the faces with indexes of $i + \frac{1}{2}$ or 
$i-\frac{1}{2}$. This project will initially focus on this 1-D configuration.
Usually the code  is 3-D,  with channels connecting to each other in two more 
dimensions.

\begin{figure}[!h]
	\centering
	\includegraphics[width=0.30\textwidth]{images/CTF-Cells}
	\caption{The finite volume structure for CTF}
	\label{fig:CTF-Cells}
\end{figure}

%The single phase Euler partial differential equations for mass
%\eqref{eq:pde_mass}, momentum \eqref{eq:pde_momentum}, and energy
%\eqref{eq:pde_energy} correspond to the unknown variables pressure $P$,
%velocity $u$, and enthalpy $h$. Density $\rho$ is related through an equation
%of state \cite{Cooper2007}. The time rate of change of density $\partial
%\rho}{\partial t}$
For the single phase formulation of the Euler equations, the unknown variables
are pressure $P$, velocity $u$, enthalpy $h$, and density $\rho$. To solve for
these variables, three conservation equations and one equation of state are
used \cite{Cooper2007}. The conservation of mass given in equation
\ref{eq:pde_mass} is the most basic with the rate of change in density equal to
the advection from the upwinded cell. The conservation of momentum in equation \ref{eq:pde_momentum}
balances the time rate of change of momentum, the advection of momentum from
adjacent cells, the gradient of pressure, and body forces. The conservation of
energy equation \ref{eq:pde_energy} contains two temporal terms; the time
rate of change of the enthalpy, and the time rate of change of the pressure.
These temporal terms are balanced against the advection of the enthalpy. 
For this work, the equation of state will be assumed to be approximately linear.
This assumption proves valid due to the small changes in pressure and
enthalpy as described later.
    
    \begin{equation}
    	\label{eq:pde_mass}
    	\frac{ \partial \rho}{\partial t} + \bigtriangledown \rho u = 0
    \end{equation}
    
    \begin{equation}
    	\label{eq:pde_momentum}
    	\frac{ \partial \rho u}{\partial t} + \bigtriangledown \rho u^{2} +
    	\bigtriangledown P - \rho g  = 0
    \end{equation}
    
    \begin{equation}
    	\label{eq:pde_energy}
    	\frac{ \partial \rho h}{\partial t} -
    	\frac{ \partial  P}{\partial t} + 
    	\bigtriangledown ( \rho  u h )
    	= 0
    \end{equation}

\subsection{Residual Formulation and Jacobian Construction}

	A residual is simply the difference between the value at some future iteration
    $k+1$ and the value at the current iteration $k$. Currently in CTF, the
    future iteration is taken to be the next time step $n+1$ and the
    current iteration is the current time $n$. The residual can be expressed for
    desired variables or conservation equations. For example, the residual for
    density, $\delta \rho_{i}$, is the difference between iterate levels $k+1$
    and $k$, $\rho^{k+1}_{i} - \rho^{k}_{i}$. The residuals for the equations
    are determined by substituting the residuals into the discretized equations,
    which should effectively change all $n+1$ into $k$. Each cell will have
    three residual variables and three residual equations. For the entire
    solution, we will then have a residual variable array $\delta X$, and a
    residual function array $F(X)$ which defines a linear system $J \delta X = -
    F(X)$. The Jacobian matrix is defined as the derivative of each response of
    the function $F_{j}$ with respect to each variable $X_{i}$.
    The derivative can be calculated numerically as shown by equation
    \eqref{eq:jac_numerical} where $\epsilon$ is a small numerical value.
    Since the system is considered linear, the approximation of the
    Jacobian matrix in this manner is accurate. As a check, the
    numerically computed Jacobian matrix was reduced to a pressure matrix using
    guassian simulation and compared to the analytical pressure matrix from CTF.
    Each of the entries of the Jacobian matrix appeared to match to nearly
    within machine precision. 
    
    \begin{equation}
    	\label{eq:jac_numerical}
    	J_{i,j}=\frac{ \partial F_{j}(X)}{\partial X_{i}}
    	      \approx \frac{F_{j}(X_{i}+\epsilon)-F_{j}(X)}{\epsilon}
    \end{equation}
    
	To build the Jacobian matrix, an object oriented class was created that
    contains three arrays; an array that points to the residual functions, an
    array that points to the position within a target variable array, and an
    array that has the index that the function is to be evaluated at. These
    lists can be appended in any order, but they have to be appended
    simultaneously such that variables and functions correspond with each
    other. To construct the Jacobian matrix, the residual function and
    residual variable arrays can each be looped over to numerically build the
    Jacobian matrix as seen in figure \ref{fig:Jacobian_Setup}. 
    
    \begin{figure}[!h]
    	\centering
    	\includegraphics[width=0.40\textwidth]{images/Jacobian_Setup}
    	\caption{Structure of the Jacobian matrix for single phase liquid}
    	\label{fig:Jacobian_Setup}
    \end{figure}

\section{Isokinetic Sine Wave Advection}

The procedures that can be used for code verification, from least to most
rigorous, include: expert judgment, error quantification,
consistency/convergence, and order of accuracy \cite{Oberkampf2008}. For this
work, the Richardson extrapolation will be used to check for convergence and
order of accuracy of the error in space and time. The error should converge to
zero, and the order of accuracy should converge to the values obtained through
the modified equation analysis \cite{Villatoro1999} at the end of this section.
 
\subsection{Problem Setup}

%Round off error, iterative convergence error.
%Discretization error?? (Check with)

To obtain an analytical solution for a subchannel code, typically the method
of manufactured solutions \cite{Knupp2000} is needed. To readily obtain an
analytical solution and isolate only the mass and energy conservation equations,
several simplifications to the verification problem are made. Only one channel
is considered to make the problem 1-D. In order to make the problem perfectly
isobaric and isokinetic, grid spacer losses, frictional losses, and gravity head
losses are set to zero representing a smooth horizontal pipe. Small fluctuations
in pressure and velocity may still occur due to the assumption that the EOS is
linear. The channel geometry and operating conditions approximate a standard PWR
as shown in table \ref{table:parameters}. The inlet of the channel has a
constant velocity with a fluctuating enthalpy that corresponds to be near the
standard PWR rod bundle coolant channel inlet conditions. The problem will aslo
have constant axial spacing and time step size. The length of the
transient was defined to be quadruple the time needed for the liquid at the
inlet to advect to the outlet. The frequency of the sine wave was defined to
generate a full period of a spatial wave across the length of the channel. With
these simplifications, the method of manufacturing solutions is unnecessary
since the known solutions are simply the advection of the transinet inlet
conditions. The functions for the etnhalpy $h$ and mass flow rate, $\dot{m}$,
are given in equations \ref{eq:Sine_Wave:h} and \ref{eq:Sine_Wave:m_dot} where
$x$ is the length from the inlet and $t$ is the simulated time. The functions
smoothly transition to the initial condition of a straight line across the
domain. The enthalpy and mass flow rate vary proportionally to the density such
that an isokinetic boundary condition is created at the inlet. While these
simplifications do not model a realistic problem, they appropriately isolate the
1-D single phase mass and energy conservation equations for the purpose of
verification. 

\begin{table}[h]
\center
\caption{Problem Parameters}
\label{table:parameters}
\begin{tabular}{|c|c|c|c|}
\hline
Parameter	&	Symbol	&	Value	&	Unit	\\ \hline
Axial Length	&	$L$	&	3.6586	&	$m$	\\ \hline
Channel Area	&	$A_{ch}$	&	4.94E-005	&	$m^{2}$	\\ \hline
Wetted Perimeter	&	$P_{w}$	&	1.49E-002	&	$m$	\\ \hline
Velocity	&	$V_{o}$	&	7.35	&	$\frac{m}{s}$	\\ \hline
Pressure	&	$P_{o}$	&	155.00	&	bar	\\ \hline
Temperature 1	&	$T_{1}$	&	290.00	&	$^{\circ}$C	\\ \hline
Temperature 2	&	$T_{2}$	&	295.00	&	$^{\circ}$C	\\ \hline
Enthalpy 1	&	$h_{1}$	&	1306.3	&	$\frac{kJ}{kg}$	\\ \hline
Enthalpy 2	&	$h_{2}$	&	1310.9	&	$\frac{kJ}{kg}$	\\ \hline
Mass Flow Rate 1	&	$\dot{m}_{1}$	&	0.2707	&	$\frac{kg}{s}$	\\ \hline
Mass Flow Rate 2	&	$\dot{m}_{2}$	&	0.2672	&	$\frac{kg}{s}$	\\ \hline
Final Time	&	$t_{f}$	&	2.00	&	sec	\\ \hline
Wave Frequency	&	$\omega$	&	1.00	&	Hz	\\ \hline
\end{tabular}
\end{table}


%$ \frac{\partial ^{2} h}{\partial t^{2}} = - \frac{1}{2}  \omega^{2}
%(h_{1}-h_{2}) cos ( \omega ( t - \frac{x}{V_{o}} ))) $
%
%$ \frac{\partial ^{2} h}{\partial x^{2}} = - \frac{1}{2} 
%\frac{\omega^{2}}{V^{2}_{o}} (h_{1}-h_{2}) cos ( \omega ( t - \frac{x}{V_{o}}
%))) $
%
%$ -\frac{\omega^{2}}{4} (h_{1}-h_{2})( \Delta t  - \frac{ \Delta x }{V_{o}} )
%(cos ( \omega ( t + \frac{x}{V_{o}} )))  $
%
%$ -\frac{x}{L} \frac{\omega^{2}}{4} (h_{1}-h_{2})( \Delta t  - \frac{ \Delta x
%}{V_{o}} ) (cos ( \omega ( t + \frac{x}{V_{o}} )))  $

\begin{equation}
	\label{eq:Sine_Wave:h}
	h(x,t) = \frac{1}{2} \left( 
			(h_{1}+h_{2}) + (h_{1}-h_{2}) cos\left(
				\omega \left( t - \frac{x}{V_{o}} \right)
				\right)
			\right)
\end{equation}

\begin{equation}
	\label{eq:Sine_Wave:m_dot}
	\dot{m}(x,t) = \frac{1}{2} \left( 
			(\dot{m}_{1}+\dot{m}_{2}) + (\dot{m}_{1}-\dot{m}_{2}) cos\left(
				\omega \left( t - \frac{x}{V_{o}} \right)
				\right)
			\right)
\end{equation}

The comparison between the data table and the output in CTF are shown for
enthalpy and mass flow rate in figures \ref{fig:Inlet_h} and
\ref{fig:Inlet_m_dot}, respectively. The CTF output was read from the high
precision VTK data files at each point in time, which omitted the actual ghost
cell where these values were applied. The CTF values are located at the nearest
node to the inlet, and will experience small amounts of numerical diffusion. For
large mesh sizes, this discrepancy is negligible as can be seen by the
overlapping profiles in figures \ref{fig:Inlet_h} and \ref{fig:Inlet_m_dot}.

\begin{figure}[!h]
	\centering
	\includegraphics[width=0.60\textwidth]{images/Code_Verification/run_00_00/residual/results/Inlet_h}
	\caption{Enthalpy Near the Inlet and the Analytical Solution}
	\label{fig:Inlet_h}
\end{figure}

\begin{figure}[!h]
	\centering
	\includegraphics[width=0.60\textwidth]{images/Code_Verification/run_00_00/residual/results/Inlet_m_dot}
	\caption{Density Near the Inlet and the Analytical Solution}
	\label{fig:Inlet_m_dot}
\end{figure}

% For the original version of CTF, there is a small discrepancy in the way the
% density is calculated at the inlet that causes the velocity to be non-constant.
%as shown in figure \ref{fig:Inlet_vel}.
 The pressure and the velocity fluctuate by less than $0.25\%$ during the
 simulation due to approximating the EOS as a linear function. This is considered small for
 this problem and should not greatly affect the order of accuracy of the error. 
 The VTK output files allow for a high level of precision, reducing
 round off error in the output during the post processing.

%\subsection{Code Convergence}
%
%The current version of CTF uses global code convergence criteria that are
%used to estimate that are used to estimate the storage of mass and energy in the system. The
%transient values of these criteria are shown in figure \ref{fig:Code_Convergence:Original} for the original version of
%CTF simulating the verification problem. The fluid energy storage is the percent
%energy stored in the fluid during one time step. The mass storage is the
%percentage of mass stored in the model during one time step. The global energy
%balance is the percent difference between the energy leaving and entering the
%system relative to the energy entering the system. The global mass balance is the
%percent difference between the mas leaving and entering the system relative to
%the energy entering the sytem. The flat profile for the mass storage term means
%that the fluid that was initially at the inlet as advected out of the system.
%After this point the sine wave can be considered to be fully developed, and
%numerical analysis should only be done after this point in time for the
%original version of the code.
%
%\begin{figure}[!h]
%	\centering
%	\includegraphics[width=0.50\textwidth]{images/Code_Verification/run_00_00/original/results/Convergence_Plot}
%	\caption{Code Convergence Criteria for the Original Version of CTF}
%	\label{fig:Code_Convergence:Original}
%\end{figure}
%
%\begin{figure}[!h]
%	\centering
%	\includegraphics[width=0.50\textwidth]{images/Code_Verification/run_00_00/residual/results/Residuals_Plot}
%	\caption{Summation of the Residuals for the Residual Version of CTF}
%	\label{fig:Residuals_Plot}
%\end{figure}
%
%The residual formulation prints out the summation of the equation residuals
%across the domain to an output file at the end of each time step and can be seen
%in figure \ref{fig:Residuals_Plot}. Unlike the previous convergence criteria,
%these are not relative values but the l2 normaliztion of each of the fluid
%conservation equations across the domain. The mass equation residual is in units
%of $\frac{kg}{m^{3}s}$. The energy equation residual is in units of
%$\frac{kW}{m^{3}}$. The momentum residual is in units of
%$\frac{kg}{m^{2}s^{2}}$. The flat profile of the mass and energy residuals shows
%that the fluid that was initially at the inlet as advected out of the system.
%After this point the sine wave can be considered to be fully developed, and
%numerical analysis should only be done after this point in time for the residual
%formulation. The 

%\subsection{Error Quantification}
%
%\begin{figure}[!h]
%	\centering
%	\includegraphics[width=1.00\textwidth]{images/Code_Verification/run_00_01/residual/results/tmp/h_0200.png}
%	\caption{Summation of the residuals for the residual version of CTF}
%	\label{fig:Residuals_Plot}
%\end{figure}
%
%\begin{figure}[!h]
%	\centering
%	\includegraphics[width=0.55\textwidth]{images/Code_Verification/run_00_01/residual/results/tmp/h_0200}
%	\caption{Summation of the residuals for the residual version of CTF}
%	\label{fig:Residuals_Plot}
%\end{figure}

\subsection{Modified Equation Analysis}
    
    The order of accuracy in time and space can be analytically determined for
    this problem through a modified equation analysis. Because the velocity is
    constant, it can be pulled out of the spatial derivative as shown in
    equation \ref{eq:isokinetic_start}. Using upwinding, the finite difference
    can be written to look like equation \ref{eq:mass_isok_fd}.  A second order
    Taylor series approximation can  be used for $\rho_{i}^{n+1}$  and
    $\rho_{i-1}^{n}$ as shown in equations \ref{eq:rho_taylor_series_time} and
    \ref{eq:rho_taylor_series_space} respectively. The higher order terms
    ($O(\Delta x^{2},\Delta t^{2} )$) are not taken into account for this
    approximation. The Taylor series approximations can then be substituted into
    \ref{eq:mass_isok_fd} to yield \ref{eq:MEA_start}. This is the beginning of
    the modified equation analysis. The goal will be to isolate the original PDE
    and define the truncation error.
    
    \begin{equation}
    	\label{eq:isokinetic_start}
    	\frac{\partial \rho}{\partial t} + U_{0} \frac{\partial \rho}{\partial x} = 0
    \end{equation}
    
    \begin{equation}
    	\label{eq:mass_isok_fd}
    	\frac{ \rho_{i}^{n+1} - \rho_{i}^{n} }{\Delta t} 
    	+ U_{0} \frac{\rho_{i}^{n} - \rho_{i-1}^{n}}{\Delta x} = 0
    \end{equation}
    
    \begin{equation}
    	\label{eq:rho_taylor_series_time}
    	\rho_{i}^{n+1} =  \rho_{i}^{n} + 
    	\frac{\partial \rho}{\partial t} \Delta t +
    	\frac{1}{2} \frac{\partial^2 \rho}{\partial t^2} \Delta t^2 + O(\Delta t^{3})
    \end{equation}
    
    \begin{equation}
    	\label{eq:rho_taylor_series_space}
    	\rho_{i-1}^{n} =  \rho_{i}^{n} - 
    	\frac{\partial \rho}{\partial x} \Delta x +
    	\frac{1}{2} \frac{\partial^2 \rho}{\partial x^2} \Delta x^2 + O(\Delta x^{3})
    \end{equation}
    
    The lengthy equation \ref{eq:MEA_start} can be reduced to equation
    \ref{eq:MEA_p0} since the $\rho_{i}^{n}$ terms subtract out and the $\Delta
    t$ and $\Delta x$ terms in the denominator cancel out. This reduced equation
    can be re-written into equation \ref{eq:MEA_p1}, with the original PDE
    followed by the truncation terms. 
    
    \begin{equation}
    	\label{eq:MEA_start}
    	\frac{ \left( \rho_{i}^{n} + \frac{\partial \rho}{\partial t} \Delta t +
    	\frac{1}{2} \frac{\partial^2 \rho}{\partial t^2} \Delta t^2 \right)-\rho_{i}^{n} }{\Delta t} 
    	+ U_{0} \frac{\rho_{i}^{n} - \left( \rho_{i}^{n} -  \frac{\partial \rho}{\partial x} \Delta x + 
    	\frac{1}{2} \frac{\partial^2 \rho}{\partial x^2} \Delta x^2 \right)}{\Delta x} 
    	+ O(\Delta x^{2},\Delta t^{2}) 
    	= 0
    \end{equation}
    
    \begin{equation}
    	\label{eq:MEA_p0}
    	 \frac{\partial \rho}{\partial t}  + \frac{1}{2} \frac{\partial^2 \rho}{\partial t^2} \Delta t +
    	 U_{0} \left(   \frac{\partial \rho}{\partial x}  - \frac{1}{2} \frac{\partial^2 \rho}{\partial x^2} \Delta x \right) 
    	 + O(\Delta x^{2},\Delta t^{2}) 
    	 = 0
    \end{equation}
    
    The terms to the right of the original PDE are the first order accurate
    truncation terms. Notice how the truncation error is  dependent on both the
    on the second derivatives of density with respect to space and time, and on
    the numerical spacing $\Delta t$ and $\Delta x$. Since the truncation
    error is linearly dependent on $\Delta t$ and $\Delta x$, the order of
    accuracy is 1 with respect to time and space. 
    
    \begin{equation}
    	\label{eq:MEA_p1}
    	 \frac{\partial \rho}{\partial t}  +  U_{0} \frac{\partial \rho}{\partial x} + 
    	 \frac{1}{2} \frac{\partial^2 \rho}{\partial t^2} \Delta t -
    	   U_{0}  \frac{1}{2} \frac{\partial^2 \rho}{\partial x^2} \Delta x  
    	   + O(\Delta x^{2},\Delta t^{2}) = 0 
    \end{equation} 
    
    When the energy equation undergoes a similar modified equation analysis, the
    order of accuracy is also 1 for time and space. The momentum conservation
    equation does not apply for this problem since the velocity is
    constant.
    

\section{Richardson Extrapolation}

The Richardson extrapolation was performed by refining the spatial and temporal
step sizes by a factor of 2 for a set number of times. The spatial and temporal
studies are refined separately in their own study in order to isolate the
spatial and temporal affects on the solution. The generation of the inputs,
running of the codes, and analysis of the output were automated with a python
script in order to reduce user input errors and increase repeatability. The
computational resources for the spatial study were much higher than the temporal
study due to the need to keep the courant number below 0.500. To keep the
computational resources needed to perform this analysis reasonable, fewer
spatial refinements were performed compared to the temporal analysis.

\subsection{Convergence of Error}

The difference between iterations was computed at each time step and spatial
location for each quantity of interest. This difference is considered as the
error between each iteration. For the spatial refinement, the lower iterate
values were numerically integrated to match the shape of the initial domain. The
errors were then summed over the entire domain to yield a total error for each
variable. The total error for density is plotted in figures
\ref{fig:Temporal:Diff_rho} and \ref{fig:Spatial:Diff_rho} as a function of
temporal and spatial step size.

\begin{figure}[!h]
	\centering
	\includegraphics[width=0.65\textwidth]{images/Temporal_Study/Difference_rho}
	\caption{Difference Between Successive Temporal Refinements for Density}
	\label{fig:Temporal:Diff_rho}
\end{figure} 

The data points were chosen to be inside of the asymptotic range as shown by
the good power fit with an exponent near 1. The power fit shows that as the
temporal and spatial step sizes are reduced, the numerical error approaches
zero. The residual formulation and original versions of CTF show good agreement
with each other.

\begin{figure}[!h]
	\centering
	\includegraphics[width=0.65\textwidth]{images/Spatial_Study/Difference_rho}
	\caption{Difference Between Successive Spatial Refinements for Density}
	\label{fig:Spatial:Diff_rho}
\end{figure} 

\subsection{Order of Accuracy}

The order of accuracy for this verification problem is first order as shown by
the modified equation analysis. This can be considered to be the exponent on
the power fits as seen in figures \ref{fig:Temporal:Diff_rho}. However the order
of accuracy $p$ can be calculated by using equation \ref{eq:OOA} where $f_{1}$,
$f_{2}$, $f_{3}$ are consecutive levels within the same Richardson extrapolation
study. The refinement factor, $R$, has the constant value of 2 for both the
spatial and temporal studies.

\begin{equation}
	\label{eq:OOA}
	p= \frac{
	      	ln \left(
	      	\frac{f_{3}-f_{2}}{f_{2}-f_{1}}
	      	\right)
	    }{ln(R)}
\end{equation}

The order of accuracy for density and enthalpy are presented for the temporal
analysis and spatial analysis in figures \ref{fig:Temporal:OOA} and
\ref{fig:Spatial:OOA} respectively. The temporal order of accuracy is well
within the asymptotic range for the whole analysis, and moves closer to 1.0 with
decreasing time step size. The spatial order of accuracy is a slightly outside
the asymptotic range, but approaches an order of accuracy of 1.0 with
decreasing mesh size. 

\begin{figure}[!h]
	\centering
	\includegraphics[width=0.65\textwidth]{images/Temporal_Study/Order_Of_Accuracy_Summary}
	\caption{Temporal Order of Accuracy}
	\label{fig:Temporal:OOA}
\end{figure}

\begin{figure}[!h]
	\centering
	\includegraphics[width=0.65\textwidth]{images/Spatial_Study/Order_Of_Accuracy_Summary}
	\caption{Spatial Order of Accuracy}
	\label{fig:Spatial:OOA}
\end{figure}

The slight differences between the original version of CTF and the residual
formulation might be due to the different solution methods and back substitution
of variables. Despite the small differences, both versions of the code exhibit
order of accuracies very close to 1 which was obtained through the modified
equation analysis. One advantage of the residual formulation is that different
numerical techniques can be readily implemented that could match higher orders
of accuracies from other codes \cite{Merroun2009}. 

%\section{Parameter Study}
%
%Tables and figures of varying the results.
%
%\subsection{Changes in $\Delta$T}
%
%This changes the amplitude of the displacement
%
%\subsection{Changes in frequency}
%
%This changes the frequency of the displacement
%
%\subsection{Changes in Mass Flow Rate}
%
%\section{Computational Time}
%
%The computational time of the two methods for different computational sizes.
%Compare the semi-implicit and fully implicit methods at 0.5 , 1.0, and 2.0 CFL. 

\section{Conclusions}

The residual formulation of CTF allows for a numerical computation of the
multivariable Jacobian matrix compared to the original analytical derivation of
a pressure matrix. The 1-D isokinetic single phase liquid verification problem
is a good verification problem due its isolation of the order
of accuracies through modified equation analysis. The discretization error for
both versions of the code converged to zero with decreasing time step and axial
mesh size. The order of accuracy for the temporal and spatial refinements
matched very closely with the modified equation analysis for both codes. For all
of these data points, the residual formulation of the code showed discretization
errors that were very close with the original version of the code. Future work
should compare the numerical error obtained in the code to the analytical
error predicted by the modified equation analysis using the derivatives of the
known solutions. While within the asymptotic range, the first order accurate
analytical error should almost exactly match the error from the code.

%%%%%%%%%%%%%%%%%%%%%%%%%%%%%%%%%%%%%%%%%%%%%%%%%%%%%%%%%%%%%%%%%%%%%
\section{Acknowledgments}

This work has been supported by the Consortium for Advanced Simulation of Light
water reactors, an Energy Innovation Hub for Modeling and
Simulation of Nuclear Reactors under U.S. Department of Energy Contract No.
DE-AC05-00OR22725.

%%%%%%%%%%%%%%%%%%%%%%%%%%%%%%%%%%%%%%%%%%%%%%%%%%%%%%%%%%%%%%%%%%%%%
\setlength{\baselineskip}{12pt}

\bibliographystyle{mc2015}
\bibliography{references}

%%%%%%%%%%%%%%%%%%%%%%%%%%%%%%%%%%%%%%%%%%%%%%%%%%%%%%%%%%%%%%%%%%%%%

%\appendix
%\section{}


\end{document}
